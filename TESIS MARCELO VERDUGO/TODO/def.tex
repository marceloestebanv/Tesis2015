\chapter{Definición del Problema y Análisis}
\label{definicion}

\section{Definición del problema} \label{prob}

\vspace{1mm}
\normalsize

Tal como se mencionó en párrafos anteriores, los profesionales del PPF Aitué utilizan herramientas de evaluación, las cuales logran capturan  el funcionamiento familiar.
Para llenar esta escala el profesional realiza visitas domiciliarias y observaciones, donde dicha información permite al profesional formarse un juicio sobre las características del funcionamiento familiar actual. Posteriormente la herramienta permite ordenar esta información y además exige la asignación de puntajes a las diferentes dimensiones que cubre. 
Esta escala (en papel) es llenada manualmente por el profesional lo que permite obtener un puntaje para cada dimensión evaluada. Esta puntuación es analizada para ver si es necesario intervenir o no en la familia evaluada, con el fin de prevenir maltratos infantiles y negligencia parental.\cite{VALENCIA2010}
De acuerdo a lo anterior, se detectan los siguientes problemas:

\begin{itemize}
	\item Falta de un sistema que automatice el proceso de la asignación de puntajes por parte del profesional, para que este, posteriormente pueda realizar la apreciación familiar.
	\item Falta de disposición de información digital de las apreciaciones familiares
	\item Falta de información útil y no trivial para apoyar la toma de decisiones del profesional  
\end{itemize}

\vspace{1mm}

\section{Objetivos}

\vspace{1mm}

\normalsize

En esta sección se explicarán los objetivos generales y los objetivos específicos para lograr el desarrollo del sistema propuesto en el Trabajo de Título.

\subsection{Objetivo General}

El objetivo de este trabajo de título es desarrollar un sistema que automatice la herramienta de apreciación NCFAS y que además, proporcione información útil y no trivial cuando un profesional del PPF Aitué realice una apreciación familiar.

\subsection{Objetivos Específicos}


Para cumplir con el objetivo general es necesario cumplir los siguientes
objetivos específicos:
\begin{itemize}
	
	\item Realizar simulación de datos.
	
	\item Analizar y comparar diferentes técnicas de minería de datos y estadística descriptiva.
	
	\item	Implementar las técnicas de Minería de Datos seleccionada. 
	
	\item	Generar reportes según las técnicas de estadística descriptiva seleccionada. 
	
	\item	Detectar patrones dentro de los descriptores de la herramienta NCFAS, utilizando Minería de Datos. 
	
	
\end{itemize}


\section{Sistema Actual}
\vspace{1mm}
\normalsize

El sistema actual del PPF Aitué para la apreciación familiar utilizando la herramienta NCFAS se puede describir bajo los siguientes pasos:

\begin{itemize}
	\item \textbf{1:} El NNA es derivado al PPF Aitué
	\item \textbf{2:} El asistente social a cargo recopila información necesaria de la familia y del NNA 
	\item \textbf{3:} El profesional utiliza la información recopilada para evaluar la familia utilizando la herramienta NCFAS en papel
	\item \textbf{4:} Obtiene los resultados de la herramienta NCFAS generando un informe
	\item \textbf{5:} De acuerdo al informe se determina el plan de intervención que necesita la familia y el NNA
\end{itemize}

En la Figura \ref{Figura9} se ejemplifica el funcionamiento del sistema actual en el PPF Aitué.



\begin{figure}[htb]
	\label{Figura9}
	\begin{center}
		\includegraphics[scale=0.5]{imagenes/diagramancfas.png}
	\end{center}
	\caption{Diagrama del Sistema Actual.}
\end{figure}





\section{Origen del Cambio}
\vspace{1mm}
\normalsize

Con el objetivo de automatizar el sistema actual se propone crear un sistema que permitirá al profesional del PPF Aitué automatizar el proceso de asignación de puntaje por medio de la herramienta NCFAS. Este sistema, además, dispondrá de diferentes módulos (estadística descriptiva y minería de datos), los cuales entregarán al profesional información útil y no trivial para apoyarlo en la toma de decisiones con respecto a la prevención de maltratos infantiles y la negligencia parental. 
Cabe mencionar que para la realización de este sistema sólo se utilizarán datos simulados validados por los profesionales. 

\subsection{Producto}
\vspace{1mm}
\normalsize

Los productos que se obtendrán tras el desarrollo del presente Trabajo de Título son los siguientes:
\begin{itemize}
	\item Un sistema de minería de datos para el apoyo de los profesionales que trabajan con  NCFAS en el PPF Aitué
	\item Manual de usuario
	\item Documentación del análisis del requerimientos ( Especificación de requerimientos) y documentación del diseño para la escalabilidad del sistema
	\item Informe final del Trabajo de Título donde se encontrará detallado el desarrollo del sistema.
\end{itemize}

\subsection{Impacto}
\vspace{1mm}
\normalsize

Con el apoyo de este sistema el profesional, al realizar una nueva apreciación familiar podrá:
\begin{itemize}
	\item Realizar el proceso de manera más eficiente.
	\item Encontrar información útil y no trivial para obtener apoyo a la hora de tomar decisiones, lo cual permitirá realizar un mejor proceso de prevención de maltratos y negligencia parental.
	\item Mejorar el cumplimiento de los profesionales con respecto plazos establecidos para la apreciación familiar. 
\end{itemize}

\section{Especificación de Requerimientos}
\vspace{1mm}
\normalsize

\subsection{Definición de requerimientos}
\vspace{1mm}
\normalsize
En esta sección se presentan los requerimientos para la realización del sistema de apoyo para los profesionales del PPF Aitué. Estos requerimientos se clasificarán en requerimientos funcionales y no funcionales. 

\subsubsection{Requerimientos funcionales}
A continuación se presentan los requerimientos funcionales los cuales se identificarán por la sigla RF.

\begin{itemize}
	\item \textbf{RF1:} Creación de usuarios del sistema con sus respectivos perfiles 
	\item \textbf{RF2:} Digitalización de la herramienta NCFAS
	\item \textbf{RF3:} Creación de un repositorio de las apreciaciones realizadas por los profesionales
	\item \textbf{RF4:} Creación de un módulo de minería de datos
	\item \textbf{RF5:} Creación de un módulo de comparación entre las apreciaciones de ingreso y egreso de la familia
	\item \textbf{RF6:} Creación de un módulo de estadística descriptiva
	\item \textbf{RF7:} Creación de un módulo de búsqueda de las apreciaciones realizadas 
\end{itemize}

\subsubsection{Requerimientos no funcionales}

A continuación se presentan los requerimientos funcionales los cuales se identificarán por la sigla RNF.

\begin{itemize}
	\item \textbf{RNF8:} El sistema debe proveer al usuario de una interfaz usable
	\item \textbf{RNF9:} El sistema debe soportar S.O Windows 7 en adelante 
	\item \textbf{RNF10:} El sistema debe proveer seguridad, ya que se trabajará con datos sensibles
	\item \textbf{RNF11:} El sistema debe proveer ayuda para agilizar el proceso de apreciación ( Mostrar descriptores de los ítems de NCFAS)
	\item \textbf{RNF12:} La interfaz debe ser amigable y contener imágenes representativas para la NCFAS
\end{itemize}

En la Tabla \ref{tablareq} se define la sigla, clasificación y priorización de los requerimientos identificados. A continuación se definen los parámetros utilizados: 

\begin{itemize}
	\item \textbf{Funcional:} El requerimiento expresa la naturaleza del funcionamiento de la plataforma
	\item \textbf{No funcional:} El requerimiento expresa las restricciones de funcionamiento de la plataforma
	\item \textbf{Obligatorio:} El requerimiento se debe desarrollar para el funcionamiento del sistema
	\item \textbf{Necesario:} El requerimiento es requerido pero no imprescindible
	\item \textbf{Prescindible:} El requerimiento se puede dejar de desarrollar y el sistema de igual forma podrá funcionar
\end{itemize}

\begin{table}[h]
\centering
\label{tablareq}
\begin{tabular}{|c|c|l|c|}
	\hline \textbf{Requerimiento} & \textbf{Sigla} & \textbf{Calificación} & \textbf{Prioridad}\\
	\hline 1 & RF1 & Funcional & Obligatorio \\ 
	\hline 2 & RF2 & Funcional & Obligatorio \\ 
	\hline 3 & RF3 & Funcional & Obligatorio \\ 
	\hline 4 & RF4 & Funcional & Obligatorio \\ 
	\hline 5 & RF5 & Funcional & Obligatorio \\ 
	\hline 6 & RNF6 & No Funcional & Obligatorio \\ 
	\hline 7 & RNF7 & No Funcional & Prescindible \\ 
	\hline 8 & RNF8 & No Funcional & Obligatorio \\ 
	\hline 9 & RNF9 & No Funcional & Necesario \\ 
	\hline 10 & RNF10 & No Funcional & Necesario \\ 
	\hline 11 & RNF11 & No Funcional & Prescindible \\ 
	\hline 12 & RNF12 & No Funcional & Prescindible \\
	\hline 
\end{tabular} 
\caption{Tabla Requerimientos del Sistema}
\end{table}

\subsection{Definición de usuarios y tareas del sistema}
El Sistema propuesto diferencia a 2 tipos de usuario, estos son los siguientes: 

\begin{itemize}
	\item \textbf{Administrador:} Usuario con los privilegios necesarios para manejar todos los datos y apreciaciones almacenadas en el sistema.
	\item \textbf{Profesional a cargo:} Usuario encargado de realizar las apreciaciones, además puede utilizar todas las herramientas del sistema (Visualizar información útil y no trivial, buscar apreciaciones anteriores, comparar apreciaciones, etc.)
\end{itemize}
	Para definir el dominio del sistema se definen los siguientes parámetros: 
	
	\begin{itemize}
		\item \textbf{Integro:} Conocimiento total de la plataforma.
		\item \textbf{Esencial:} Conocimiento completo de los módulos utilizados por el profesional a cargo.
	\end{itemize}
	
Además se considera la variable frecuencia de interacción cuyos parámetros son los siguientes: 

\begin{itemize}
	\item \textbf{Alta:} Utilización diaria de la plataforma.
	\item \textbf{Baja:} Utilización mensual o a mayor plazo de la plataforma.
\end{itemize} 

Finalmente se considera para cada tarea de usuario su importancia. Esta variable está determinada con los siguientes parámetros:

\begin{itemize}
	\item \textbf{Crítica:} Cuando la tarea es fundamental para el correcto funcionamiento de la plataforma.
	\item \textbf{Necesaria:} Cuando la tarea afecta sólo a algunos módulos de la plataforma.
	\item \textbf{Intrascendente:} Cuando la tarea es independiente de los demás módulos de la plataforma.
\end{itemize}

\clearpage
\newpage

En la tabla \ref{tareasuser}  se presentan los usuarios y las tareas clasificadas con los parámetros mencionados anteriormente.\\

\begin{table}[h]
	\centering
	\label{tareasuser}
\begin{tabular}{|p{2.5cm}|p{1.5cm}|p{6cm}|p{2cm}|p{2cm}|}
	\hline \textbf{Nivel de Acceso} & \textbf{Dominio del Sistema} & \textbf{Tareas} & \textbf{Importancia}  & \textbf{Frecuencia de Interacción}\\ 
	\hline Administrador & Integro &  Gestionar Usuarios del Sistema & Crítica & Baja \\
	\hline &  & Gestionar NCFAS digital & Crítica & \\
	\cline{3-4}
	Profesional a Cargo & Esencial & Visualizar información útil y no trivial & Necesaria  & Alta\\\cline{3-4} 
	&  & Buscar apreciaciones guardadas & Necesaria & \\ 
	\hline 
\end{tabular} 
\caption{Tabla que presenta las tareas de usuario clasificadas con los parámetros mencionados anteriormente.}
\end{table}

\subsection{Definición de las funciones del sistema}

En esta sección se presentan las funciones que debe ofrecer el sistema propuesto. En la tabla \ref{tablafunc} se asocia un código a cada función.

\begin{table}[h]
\centering
\label{tablafunc}
\begin{tabular}{|c|l|}
	\hline \textbf{Código} & \textbf{Función} \\
	\hline F01 & Crear usuario \\ 
	\hline F02 & Guardar usuario  \\
	\hline F03 & Modificar usuario \\ 
	\hline F04 & Eliminar usuario \\ 
	\hline F05 & Ingresar una apreciación  \\
	\hline F06 & Modificar una nueva apreciación  \\
	\hline F07 & Guardar apreciación \\
	\hline F08 & Buscar apreciaciones guardadas \\
	\hline F09 & Desplegar apreciación guardada\\
	\hline F10 & Desplegar información módulo estadística descriptiva \\ 
	\hline F11 & Desplegar información módulo minería de datos  \\ 
	\hline F12 & Comparar apreciaciones\\
	\hline 
\end{tabular} 
\caption{Tabla Funciones del Sistema}
\end{table}

\subsection{Modelo Conceptual}

En esta sección se presenta el modelo conceptual del sistema. En este se muestran las funcionalidades que realizan las entidades más importantes que componen la totalidad del sistema.
Algunas clases de la Figura \ref{Figura10} tienen palabras abreviadas para facilitar la visualización de la imagen en donde:\\

\begin{itemize}
	\item Modulo E.D representa: Modulo de Estadística Descriptiva.
	
	\item Técnicas E.D representa: Técnicas de Estadística Descriptiva.
	
	\item Módulo M. de D. representa: Módulo de Minería de Datos.
	
	\item Técnicas de M. de D. representa: Técnicas de Minería de Datos. 
\end{itemize}

\begin{figure}[htb]
	\centering
	\label{Figura10}
	\begin{center}
		\includegraphics[scale=0.4]{imagenes/Database.png}
	\end{center}
	\caption{Modelo Conceptual del Sistema.}
\end{figure}

\clearpage 	
\newpage

\subsection{Diagrama de Casos de Uso}

En esta sección se representan las funcionalidades del sistema en forma de diagrama de casos de uso. Estos diagramas serán utilizados al momento de desarrollar las diferentes funcionalidades, ya que nos permite observar la comunicación entre ellas y los actores del sistema. La primera figura representa el caso de uso de forma general, y las siguientes figuras representan los casos de uso de manera mas detallada.

\begin{figure}[htb]
	\label{Figura11}
	\begin{center}
		\includegraphics[scale=0.4]{imagenes/diagramacdu.png}
	\end{center}
	\caption{Diagrama de Caso de Uso general.}
\end{figure}

\begin{figure}[htb]
	\label{Figura12}
	\begin{center}
		\includegraphics[scale=0.4]{imagenes/CDU1.png}
	\end{center}
	\caption{Caso de Uso Gestionar Usuario.}
\end{figure}


\begin{figure}[htb]
	\label{Figur7}
	\begin{center}
		\includegraphics[scale=0.4]{imagenes/CDU2.png}
	\end{center}
	\caption{Caso de Uso NCFAS Digital.}
\end{figure}

\newpage
\clearpage

\begin{figure}[htb]
	\label{Figura13}
	\begin{center}
		\includegraphics[scale=0.5]{imagenes/CDU3.png}
	\end{center}
	\caption{Caso de Uso Visualizar Información.}
\end{figure}

\begin{figure}[htb]
	\label{Figura14}
	\begin{center}
		\includegraphics[scale=0.5]{imagenes/CDU4.png}
	\end{center}
	\caption{Caso de Uso Buscar Apreciaciones Guardadas.}
\end{figure}

\begin{figure}[htb]
	\label{Figura15}
	\begin{center}
		\includegraphics[scale=0.5]{imagenes/CDU5.png}
	\end{center}
	\caption{Caso de Uso Comparar Apreciaciones.}
\end{figure}



\subsection{Caso de Uso en Formato Expandido}
A continuación se presentan los casos de uso en formato expandido, estos explican el proceso de interacción que existe entre el usuario y el sistema detallando cada una de las funciones que tiene el sistema propuesto. 
 
\begin{table}
\centering
\begin{tabular}{|p{6cm} |p{6cm}|}
	\hline \textbf{Caso de Uso} & Crear Usuario  \\ 
	\hline \textbf{ID} & CDU1 \\ 
	\hline \textbf{Actores} & Administrador \\ 
	\hline \textbf{Tipo} & Primario \\ 
	\hline \textbf{Pre-condición} & - \\ 
	\hline \textbf{Descripción} & Permite crear los usuarios del sistema \\
	\hline \textbf{Ref. Cruzadas} & RF1 \\ 
	\hline
	\multicolumn{2}{|c|}{\textbf{Resumen}} \\
	\hline
	\multicolumn{2}{|p{12cm}|}{El administrador crea los usuarios asignando un respectivo nombre de usuario, e-mail, contraseña y perfil para que posteriormente el usuario pueda ingresar al sistema.} \\
	\hline 
\end{tabular}  

\begin{tabular}{|p{6cm}|p{6cm}|}
	
	\multicolumn{2}{|c|}{\textbf{Curso normal de eventos}} \\
	\hline \textbf{Actor} & \textbf{Sistema} \\ 
	\hline 1. Este caso de uso comienza cuando el administrador debe crear un usuario para el sistema. & 2.El sistema solicita al administrador el nombre de usuario, e-mail, contraseña y perfil del nuevo usuario.  \\ 
	3. El administrador ingresa lo solicitado. & 4. El sistema valida lo ingresado por el administrador y crea el nuevo usuario. \\
	 & 5. El sistema guarda el nuevo usuario y permite al administrador crear un nuevo usuario o bien ingresar al sistema. \\
	 \hline
	\multicolumn{2}{|c|}{\textbf{Curso alternativo de eventos}} \\
	\hline
	\multicolumn{2}{|p{12cm}|}{4. El sistema valida los datos ingresados por el administrador pero estos son válidos y vuelve al paso 3.} \\
	\hline
\end{tabular}
\caption{Tabla del caso de uso expandido de Crear Usuario}
\label{tabcdu1}
\end{table}

\newpage
\clearpage
%CASO DE USO EXPANDIDO MODIFICAR USUARIO

\clearpage
\begin{table}
	\centering
	\begin{tabular}{|p{6cm} |p{6cm}|}
		\hline \textbf{Caso de Uso} & Modificar Usuario \\ 
		\hline \textbf{ID} & CDU2 \\ 
		\hline \textbf{Actores} & Administrador \\ 
		\hline \textbf{Tipo} & Opcional \\ 
		\hline \textbf{Pre-condición} & Crear un usuario \\ 
		\hline \textbf{Descripción} & Permite modificar un usuario creado anteriormente \\
		\hline \textbf{Ref. Cruzadas} & RF1 \\ 
		\hline
		\multicolumn{2}{|c|}{\textbf{Resumen}} \\
		\hline
		\multicolumn{2}{|p{12cm}|}{El administrador modifica un usuario creado anteriormente, con la posibilidad de modificar nombre de usuario, e-mail, contraseña o perfil.} \\
		\hline 
	\end{tabular}  
	
	
	\begin{tabular}{|p{6cm}|p{6cm}|}
		
		\multicolumn{2}{|c|}{\textbf{Curso normal de eventos}} \\
		\hline \textbf{Actor} & \textbf{Sistema} \\ 
		\hline 1. Este caso de uso comienza cuando el administrador desea modificar un usuario creado anteriormente, seleccionando la opción modificar usuario. & 2. El sistema muestra los usuarios guardados. \\ 
		3. El administrador selecciona al usuario que desea modificar. & 4. El sistema muestra las opciones a modificar del usuario  \\
		5. El administrador modifica lo necesario. &  \\
		6. El administrador guarda las modificaciones. & 7. El sistema valida las modificaciones y guarda nuevamente el usuario modificado. \\
		\hline
		\multicolumn{2}{|c|}{\textbf{Curso alternativo de eventos}} \\
		\hline
		\multicolumn{2}{|p{12cm}|}{7. El sistema valida los datos ingresados por el administrador pero estos no son válidos y vuelve al paso 5. } \\
		\hline
	\end{tabular}
	\caption{Tabla del caso de uso expandido de Modificar Usuario}
	\label{tabcdu2}
\end{table}

\newpage
\clearpage
%CASO DE USO EXPANDIDO ELIMINAR USUARIO DEL SISTEMA
\clearpage
\begin{table}
	\centering
	\begin{tabular}{|p{6cm} |p{6cm}|}
		\hline \textbf{Caso de Uso} & Eliminar Usuario \\ 
		\hline \textbf{ID} & CDU3 \\ 
		\hline \textbf{Actores} & Administrador \\ 
		\hline \textbf{Tipo} & Opcional \\ 
		\hline \textbf{Pre-condición} & Crear un usuario \\ 
		\hline \textbf{Descripción} & Permite eliminar un usuario creado anteriormente \\
		\hline \textbf{Ref. Cruzadas} & RF1 \\ 
		\hline
		\multicolumn{2}{|c|}{\textbf{Resumen}} \\
		\hline
		\multicolumn{2}{|p{12cm}|}{El administrador elimina un usuario creado anteriormente.} \\
		\hline 
	\end{tabular}  
	\begin{tabular}{|p{6cm}|p{6cm}|}
		\multicolumn{2}{|c|}{\textbf{Curso normal de eventos}} \\
		\hline \textbf{Actor} & \textbf{Sistema} \\ 
		\hline 1. Este caso de uso comienza cuando el administrador desea eliminar un usuario creado anteriormente, seleccionando la opción eliminar usuario. & 2. El sistema muestra los usuarios guardados. \\ 
		3. El administrador selecciona al usuario que desea eliminar. & 4. El sistema envía una alerta para que el administrador confirme que realmente desea eliminar el usuario seleccionado.  \\
		5. El administrador confirma la eliminación. & 6. El sistema elimina al usuario. \\
		\hline
		\multicolumn{2}{|c|}{\textbf{Curso alternativo de eventos}} \\
		\hline
		\multicolumn{2}{|p{12cm}|}{5. El administrador cancela la eliminación del usuario y vuelve al paso 3. } \\
		\hline
	\end{tabular}
	\caption{Tabla del caso de uso expandido de Eliminar Usuario}
	\label{tabcdu3}
\end{table}

\newpage
\clearpage
%CASO DE USO EXPANDIDO INGRESAR NCFAS

\begin{table}
\centering
\begin{tabular}{|p{6cm} |p{6cm}|}
	\hline \textbf{Caso de Uso} & Ingresar una Nueva Apreciación \\ 
	\hline \textbf{ID} & CDU4 \\ 
	\hline \textbf{Actores} & Profesional a cargo \\ 
	\hline \textbf{Tipo} & Primario \\ 
	\hline \textbf{Pre-condición} & - \\ 
	\hline \textbf{Descripción} & Permite ingresar una apreciación familiar por medio de NCFAS Digital \\
	\hline \textbf{Ref. Cruzadas} & RF2 - RF3 - RNF11 \\ 
	\hline
	\multicolumn{2}{|c|}{\textbf{Resumen}} \\
	\hline
	\multicolumn{2}{|p{12cm}|}{El profesional a cargo luego de recopilar la información necesaria de la familia, ordena y califica la información por medio de la herramienta NCFAS Digital.} \\
	\hline 
\end{tabular}  
\begin{tabular}{|p{6cm}|p{6cm}|}
	
	\multicolumn{2}{|c|}{\textbf{Curso normal de eventos}} \\
	\hline \textbf{Actor} & \textbf{Sistema} \\ 
	\hline 1. Este caso de uso comienza cuando el profesional a cargo ingresa una nueva apreciación familiar. & 2. El sistema despliega la herramienta NCFAS Digital.  \\ 
	3. El profesional a cargo califica cada ítem, donde además puede ver los descriptores asociados a cada ítem.&  \\
	4. El profesional guarda todo su progreso. & 5. El sistema guarda y almacena la apreciación completa. \\
	\hline
	\multicolumn{2}{|c|}{\textbf{Curso alternativo de eventos}} \\
	\hline
	\multicolumn{2}{|p{12cm}|}{4. El profesional no guarda la apreciación realizada. } \\
	\hline
\end{tabular}
\caption{Tabla del caso de uso expandido de NCFAS Digital}
\label{tabcdu4}
\end{table}

\newpage
\clearpage
%CASO DE USO EXPANDIDO MODIFICAR NCFAS

\begin{table}
	\centering
	\begin{tabular}{|p{6cm} |p{6cm}|}
		\hline \textbf{Caso de Uso} & Modificar Apreciación \\ 
		\hline \textbf{ID} & CDU5 \\ 
		\hline \textbf{Actores} & Profesional a cargo \\ 
		\hline \textbf{Tipo} & Primario \\ 
		\hline \textbf{Pre-condición} & Existan apreciaciones guardadas \\ 
		\hline \textbf{Descripción} & Permite modificar una apreciación familiar por medio de NCFAS Digital \\
		\hline \textbf{Ref. Cruzadas} & RF2 - RF3 - RF7 RNF11 \\ 
		\hline
		\multicolumn{2}{|c|}{\textbf{Resumen}} \\
		\hline
		\multicolumn{2}{|p{12cm}|}{El profesional luego guardar una apreciación, este, posteriormente puede modificarla.} \\
		\hline 
	\end{tabular}  
	\begin{tabular}{|p{6cm}|p{6cm}|}
		
		\multicolumn{2}{|c|}{\textbf{Curso normal de eventos}} \\
		\hline \textbf{Actor} & \textbf{Sistema} \\ 
		\hline 1. Este caso de uso comienza cuando el profesional a cargo desea modificar una apreciación familiar. & 2.El sistema despliega las apreciaciones guardadas.  \\ 
		3. El profesional a cargo por medio del buscador de apreciaciones, busca cual apreciación desea modificar.& 4. El sistema busca la apreciación a modificar por el profesional a cargo y la muestra. \\
		5. El profesional selecciona la apreciación a modificar. & 6. El sistema despliega la apreciación para modificarla. \\
		7. El profesional modifica lo necesario y guarda los cambios. & 8. El sistema guarda la apreciación modificada. \\
		\hline
		\multicolumn{2}{|c|}{\textbf{Curso alternativo de eventos}} \\
		\hline
		\multicolumn{2}{|p{12cm}|}{7. El profesional modifica pero no guarda los cambios. } \\
		\hline
	\end{tabular}
	\caption{Tabla del caso de uso expandido de Modificar NCFAS Digital}
	\label{tabcdu5}
\end{table}

\newpage
\clearpage
%CASO DE USO EXPANDIDO VISUALIZAR INFORMACIÓN


\newpage
\clearpage
%CASO DE USO EXPANDIDO VISUALIZAR INFORMACIÓN ESTADISTICA

\begin{table}
	\centering
	\begin{tabular}{|p{6cm} |p{6cm}|}
		\hline \textbf{Caso de Uso} & Visualizar Información Estadística Descriptiva \\ 
		\hline \textbf{ID} & CDU6 \\ 
		\hline \textbf{Actores} & Profesional a cargo \\ 
		\hline \textbf{Tipo} & Opcional \\ 
		\hline \textbf{Pre-condición} & Existan apreciaciones guardadas \\ 
		\hline \textbf{Descripción} & Permite que el profesional a cargo visualice información con técnicas de Estadística Descriptiva \\
		\hline \textbf{Ref. Cruzadas} & RF4 - RF6 \\ 
		\hline
		\multicolumn{2}{|c|}{\textbf{Resumen}} \\
		\hline
		\multicolumn{2}{|p{12cm}|}{El profesional a cargo podrá visualizar la información , mediante técnicas de estadística descriptiva.} \\
		\hline 
	\end{tabular}  
	\begin{tabular}{|p{6cm}|p{6cm}|}
		\multicolumn{2}{|c|}{\textbf{Curso normal de eventos}} \\
		\hline \textbf{Actor} & \textbf{Sistema} \\ 
		\hline 1. Este caso de uso comienza cuando el profesional a cargo desea visualizar información mediante el módulo de estadística descriptiva. & 2.El sistema despliega las diferentes técnicas de estadística descriptiva para mostrar la información.  \\ 
		3. El profesional a cargo selecciona una de estas técnicas de estadística descriptiva. & 4.El sistema despliega la información.  \\ 
		\hline
		\multicolumn{2}{|c|}{\textbf{Curso alternativo de eventos}} \\
		\hline
		\multicolumn{2}{|p{12cm}|}{ - } \\
		\hline
	\end{tabular}
	\caption{Tabla del caso de uso expandido de Visualizar Información Módulo Estadística Descriptiva}
	\label{tabcdu6}
\end{table}

\newpage
\clearpage
%CASO DE USO EXPANDIDO VISUALIZAR INFORMACIÓN MINERÍA

\begin{table}
	\centering
	\begin{tabular}{|p{6cm} |p{6cm}|}
		\hline \textbf{Caso de Uso} & Visualizar Información \\ 
		\hline \textbf{ID} & CDU7 \\ 
		\hline \textbf{Actores} & Profesional a cargo \\ 
		\hline \textbf{Tipo} & Opcional \\ 
		\hline \textbf{Pre-condición} & - \\ 
		\hline \textbf{Descripción} & Permite que el profesional a cargo visualice información con técnicas de Minería de Datos \\
		\hline \textbf{Ref. Cruzadas} & RF4 - RF6 \\ 
		\hline
		\multicolumn{2}{|c|}{\textbf{Resumen}} \\
		\hline
		\multicolumn{2}{|p{12cm}|}{El profesional a cargo podrá visualizar información mediante técnicas de minería de datos.} \\
		\hline 
	\end{tabular}  
	\begin{tabular}{|p{6cm}|p{6cm}|}
		\multicolumn{2}{|c|}{\textbf{Curso normal de eventos}} \\
		\hline \textbf{Actor} & \textbf{Sistema} \\ 
			\hline 1. Este caso de uso comienza cuando el profesional a cargo desea visualizar información mediante el módulo de minería de datos. & 2.El sistema despliega las diferentes técnicas de minería de datos para encontrar información útil y no trivial.  \\ 
			3. El profesional a cargo selecciona una de estas técnicas. & 4.El sistema despliega la información.  \\ 
			\hline
			\multicolumn{2}{|c|}{\textbf{Curso alternativo de eventos}} \\
			\hline
			\multicolumn{2}{|p{12cm}|}{ - } \\
			\hline
		\end{tabular}
		\caption{Tabla del caso de uso expandido de Visualizar Información Módulo Minería de Datos}
	\label{tabcdu7}
\end{table}

\newpage
\clearpage
% CASO DE USO EXPANDIDO BUSCAR UNA APRECIACION

\begin{table}
	\centering
	\begin{tabular}{|p{6cm} |p{6cm}|}
		\hline \textbf{Caso de Uso} & Buscar NCFAS Digital \\ 
		\hline \textbf{ID} & CDU8 \\ 
		\hline \textbf{Actores} & Profesional a cargo \\ 
		\hline \textbf{Tipo} & Opcional \\ 
		\hline \textbf{Pre-condición} & EL sistema debe tener NCFAS Digitales almacenados \\ 
		\hline \textbf{Descripción} & Permite buscar una apreciación NCFAS \\
		\hline \textbf{Ref. Cruzadas} & RF2-RF7 \\ 
		\hline
		\multicolumn{2}{|c|}{\textbf{Resumen}} \\
		\hline
		\multicolumn{2}{|p{12cm}|}{El profesional a cargo podrá buscar NCFAS Digitales dentro del sistema.} \\
		\hline 
	\end{tabular}  
	\begin{tabular}{|p{6cm}|p{6cm}|}
		\multicolumn{2}{|c|}{\textbf{Curso normal de eventos}} \\
		\hline \textbf{Actor} & \textbf{Sistema} \\ 
		\hline 1. Este caso de uso comienza cuando el profesional a cargo desea buscar una apreciación familiar. & 2.El sistema despliega un buscador.  \\ 
		3. El profesional a cargo ingresa palabras claves para buscar la apreciación necesaria.& 4. El sistema despliega las NCFAS Digitales correspondientes a la búsqueda del profesional. \\
		\hline
		\multicolumn{2}{|c|}{\textbf{Curso alternativo de eventos}} \\
		\hline
		\multicolumn{2}{|p{12cm}|}{ - } \\
		\hline
	\end{tabular}
	\caption{Tabla del caso de uso expandido de Buscar NCFAS Digital}
	\label{tabcdu8}
\end{table}

\newpage
\clearpage

% CASO DE USO EXPANDIDO COMPARAR APRECIACIONES
\begin{table}
	\centering
	\begin{tabular}{|p{6cm} |p{6cm}|}
		\hline \textbf{Caso de Uso} & Comparar NCFAS Digitales \\ 
		\hline \textbf{ID} & CDU9 \\ 
		\hline \textbf{Actores} & Profesional a cargo \\ 
		\hline \textbf{Tipo} & Opcional \\ 
		\hline \textbf{Pre-condición} & EL sistema debe tener NCFAS Digitales almacenados \\ 
		\hline \textbf{Descripción} & Permite comparar dos NCFAS(Ingreso v/s egreso) \\
		\hline \textbf{Ref. Cruzadas} & RF5 \\ 
		\hline
		\multicolumn{2}{|c|}{\textbf{Resumen}} \\
		\hline
		\multicolumn{2}{|p{12cm}|}{El profesional a cargo podrá comparar NCFAS Digitales dentro del sistema.} \\
		
	\end{tabular}  
	\begin{tabular}{|p{6cm}|p{6cm}|}
		\multicolumn{2}{|c|}{\textbf{Curso normal de eventos}} \\
		\hline \textbf{Actor} & \textbf{Sistema} \\ 
		\hline 1. Este caso de uso comienza cuando el profesional a cargo desea comparar apreciaciones familiares. & 2.El sistema despliega las NCFAS que puede seleccionar para comparar (Las que tengan una apreciación de ingreso y egreso).  \\ 
		3. El profesional a cargo selecciona las NCFAS que desea comparar.& 4. El sistema despliega información acerca de los ítems que más variaron entre el NCFAS de ingreso v/s el NCFAS de egreso. \\
		\hline
		\multicolumn{2}{|c|}{\textbf{Curso alternativo de eventos}} \\
		\hline
		\multicolumn{2}{|p{12cm}|}{ - } \\
		\hline
	\end{tabular}
	\caption{Tabla del caso de uso expandido de Comparar NCFAS Digitales}
	\label{tabcdu9}
\end{table}

\newpage
\clearpage

\subsection{Diagramas de Secuencia}

Los siguientes diagrama de secuencia ilustran gráficamente la interacción entre el usuario y el sistema. \\
A continuación se presentan los diagramas de secuencia para las funciones del sistema.

\begin{figure}[htb]
	\label{dss1}
	\begin{center}
		\includegraphics[scale=0.5]{imagenes/creauser2.png}
	\end{center}
	\caption{Diagrama de secuencia: Crear Usuario.}
\end{figure}


\begin{figure}[htb]
	\label{dss2}
	\begin{center}
		\includegraphics[scale=0.5]{imagenes/modificaruser.png}
	\end{center}
	\caption{Diagrama de secuencia: Modificar Usuario.}
\end{figure}


\begin{figure}[htb]
	\label{dss3}
	\begin{center}
		\includegraphics[scale=0.5]{imagenes/eliminaruser.png}
	\end{center}
	\caption{Diagrama de secuencia: Eliminar Usuario.}
\end{figure}


\begin{figure}[htb]
	\label{dss4}
	\begin{center}
		\includegraphics[scale=0.5]{imagenes/ingresarncfas.png}
	\end{center}
	\caption{Diagrama de secuencia: Ingresar Nueva Apreciación.}
\end{figure}


\begin{figure}[htb]
	\label{dss5}
	\begin{center}
		\includegraphics[scale=0.5]{imagenes/modificarncfas.png}
	\end{center}
	\caption{Diagrama de secuencia: Modificar Apreciación.}
\end{figure}


\begin{figure}[htb]
	\label{dss6}
	\begin{center}
		\includegraphics[scale=0.5]{imagenes/buscar.png}
	\end{center}
	\caption{Diagrama de secuencia: Buscar Apreciación.}
\end{figure}

\begin{figure}[htb]
	\label{dss7}
	\begin{center}
		\includegraphics[scale=0.5]{imagenes/visualizared.png}
	\end{center}
	\caption{Diagrama de secuencia: Visualizar Información Estadística Descriptiva.}
\end{figure}


\begin{figure}[htb]
	\label{dss8}
	\begin{center}
		\includegraphics[scale=0.5]{imagenes/visualizarmdd.png}
	\end{center}
	\caption{Diagrama de secuencia: Visualizar Información Minería de Datos.}
\end{figure}

\begin{figure}[htb]
	\label{dss9}
	\begin{center}
		\includegraphics[scale=0.5]{imagenes/comparar.png}
	\end{center}
	\caption{Diagrama de secuencia: Comparar Apreciaciones.}
\end{figure}

\newpage
\clearpage

\subsection{Diagramas de Estado}

Los diagramas de estado presentan los cambios de estado del sistema según se realicen sus funcionalidades. A continuación se presentan los diagramas de estado para las funcionalidades. 

\begin{figure}[htb]
	\label{dde1}
	\begin{center}
		\includegraphics[scale=0.5]{imagenes/crearusuario.png}
	\end{center}
	\caption{Diagrama de estado: Crear Usuario.}
\end{figure}

\begin{figure}[htb]
	\label{dde2}
	\begin{center}
		\includegraphics[scale=0.5]{imagenes/ModificarUsuario.png}
	\end{center}
	\caption{Diagrama de estado: Modificar Usuario.}
\end{figure}


\begin{figure}[htb]
	\label{dde3}
	\begin{center}
		\includegraphics[scale=0.5]{imagenes/EliminarUser2.png}
	\end{center}
	\caption{Diagrama de estado: Eliminar Usuario.}
\end{figure}

\begin{figure}[htb]
	\label{dde4}
	\begin{center}
		\includegraphics[scale=0.5]{imagenes/IngresarNCFAS2.png}
	\end{center}
	\caption{Diagrama de estado: Ingresar Apreciación.}
\end{figure}


\begin{figure}[htb]
	\label{dde5}
	\begin{center}
		\includegraphics[scale=0.5]{imagenes/ModificarNCFAS2.png}
	\end{center}
	\caption{Diagrama de estado: Modificar Apreciación.}
\end{figure}

\begin{figure}[htb]
	\label{dde6}
	\begin{center}
		\includegraphics[scale=0.5]{imagenes/BuscarNCFAS.png}
	\end{center}
	\caption{Diagrama de estado: Buscar Apreciación.}
\end{figure}


\begin{figure}[htb]
	\label{dde7}
	\begin{center}
		\includegraphics[scale=0.6]{imagenes/VisualizarInfoED.png}
	\end{center}
	\caption{Diagrama de estado: Visualizar Información Estadística Descriptiva.}
\end{figure}

\begin{figure}[htb]
	\label{dde8}
	\begin{center}
		\includegraphics[scale=0.6]{imagenes/VisualizarInfoMD.png}
	\end{center}
	\caption{Diagrama de estado: Visualizar Información Minería de Datos.}
\end{figure}


\begin{figure}[htb]
	\label{dde9}
	\begin{center}
		\includegraphics[scale=0.5]{imagenes/compararncfas.png}
	\end{center}
	\caption{Diagrama de estado: Comparar Apreciaciones.}
\end{figure}

\newpage
\clearpage

\section{Conclusión}

En el Capítulo \ref{definicion} se plantea que una herramienta digital de la escala de apreciación NCFAS sería de gran utilidad para los profesionales que trabajan en el PPF Aitué. Además se llevan acabo técnicas de diseño de software para posteriormente apoyar la etapa de desarrollo del sistema propuesto. 

