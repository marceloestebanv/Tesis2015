\chapter{Introducción}
\label{intr}
\vspace{2mm}
\normalsize


Uno de los derechos fundamentales de los niños, niñas y adolescentes es que sus necesidades sean satisfechas para desarrollarse y alcanzar la madurez. 
Esta tarea, principalmente recae en los padres y cuidadores, pero además de ellos, también se ven involucrados el conjunto de la sociedad en la que se encuentran los NNA. Por lo cual es necesario que cada adulto, cada comunidad y cada Estado disponga de los cuidados, la protección y la educación que estos necesitan para llegar a la adolescencia y luego a la vida adulta, de una forma sana, constructiva y feliz\cite{REF1}. \\

Está demostrado\cite{REF2} que los trastornos del desarrollo, comportamientos agresivos y violentos, así como otras manifestaciones negativas de los NNA, tienen una estricta relación cuando estos son víctimas y testigo de violencia en el ámbito familiar. Por lo cual es necesario prevenir los malos tratos infantiles para evitar desencadenar estas conductas negativas. Por ejemplo en la región de Valparaíso, comuna de Viña del Mar, específicamente en la población Forestal, se puede observar que la población infanto juvenil se caracterizan por ser víctimas de maltrato y negligencia, agresiones verbales, físicas y/o descalificaciones de mayor o menor gravedad. Entre los niños y niñas de edad preadolescente se observa bajos niveles de autoestima y percepción de logros; recurrentes problemas conductuales, malos tratos a nivel familiar y de pares, ausentismo y riesgo de deserción escolar y bajo rendimiento académico. Entre la población adolescente es posible observar conducta sexual precoz y/o de riesgo, conflictos delictuales, uso de alcohol y/o drogas, embarazo precoz, deserción escolar, micro tráfico,  falta de proyectos de vida, entre otros \cite{REF3}.\\

Por estos motivos nacen diferentes corporaciones que tienen por labor proteger los derechos de los NNA, uno de ellos es la Corporación Comunidad la Roca que trabaja con un proyecto llamado Programas de Prevención Focalizada (PPF Aitué) que es una de las redes colaboradoras del SENAME, que se ubica en la población de Forestal Alto en Viña del Mar. Este proyecto tiene por objetivo que los niños, niñas y adolescentes fortalezcan sus recursos personales, autoestima, auto imagen y habilidades sociales. Junto con ello, que los adultos responsables cuenten con las herramientas y oportunidades para el ejercicio de una parentalidad positiva y que las familias cuenten con apoyos para favorecer la crianza y el desarrollo de los niños, niñas y adolescentes. \\

El proyecto PPF Aitué utiliza diferentes herramientas al momento de trabajar con las familias y los NNA.
Cuando los NNA son derivados al centro (Por el TRIFAM, CESFAM, colegio, etc), primero que todo se realiza un análisis de la situación actual familiar. Para esto el profesional se dirige al hogar de NNA con el fin de poder observar las condiciones en la que él y su familia conviven, además realiza entrevistas para que posteriormente con la información obtenida, el profesional logre formar un juicio del funcionamiento familiar actual.\\

Posteriormente se utiliza una herramienta llamada NCFAS la cual permite ordenar y calificar la apreciación del profesional con respecto a la familia. Con esta apreciación realizada los profesionales logran determinar cuando es necesario intervenir en la familia con el fin de prevenir la vulneración de los derechos de los NNA; o bien para prevenir vulneraciones futuras.\\
Este proceso actualmente se realiza en papel y además, los profesionales no cuentan con una herramientas automatizadas que los apoye en la toma de decisiones.