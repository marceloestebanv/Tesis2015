\documentclass[12pt,letterpaper]{report}

\marginparsep 0pt
\textwidth 6in
\topmargin 0pt
\headsep .5in
\textheight 9.2in
\voffset = 0pt
\hoffset = 0pt
\marginparwidth = 0pt \oddsidemargin = 0pt \sloppy

%Dimensiones de la p�gina
\usepackage[left=2.5cm,top=3cm,right=2.5cm,bottom=2.5cm]{geometry}
%Sangr�a
\setlength{\parindent}{1cm}

%Numeracion
\pagenumbering{arabic}

\usepackage{templateICI}
\usepackage{graphicx}
\usepackage{graphics}
\usepackage[dvips]{epsfig}
\usepackage{times}
\usepackage[latin1]{inputenc}
\usepackage[dvips]{graphicx}
\usepackage[usenames]{color}
\usepackage[spanish]{babel}
\newcommand{\ie}{i.e.}
\newcommand {\out}[1]{}
\newtheorem{definicion}{Definicion}
\usepackage{rotating}
\usepackage{multirow}
\usepackage{array}
\usepackage{longtable}
\usepackage[]{fontenc}
\usepackage{hyperref}

\renewcommand{\shorthandsspanish}{}

\addto\captionsspanish{
\def\listtablename{�ndice de tablas}
\def\tablename{Tabla}}

\sloppy

\begin{document}
\title{\textbf{NOMBRE TRABAJO TITULO}}
\author{\textbf{Nombre Alumno}}
\principaladviser{Nombre Profesor Gu�a}
\coprincipaladviser{Nombre Profesor Correferente}
\firstreader{Nombre Profesor Informante 1}

\beforepreface
\prefacesection{Resumen}
Coloque aqui un resumen de su trabajo.
%\newpage
\prefacesection{Agradecimientos}
Aqui pueden colocar sus agradecimientos.
Si han estudiado con becas es recomendable colocar los agradecimientos a las instituciones que les otorgaron las becas.
\afterpreface

%Aqui deben incluir el fuente de cada capitulo, sin su encabezado.



\chapter{Introducci�n}
\label{intr}

Esta secci�n debe presentar una introducci�n al �rea de trabajo.
Debe introducir terminolog�a b�sica del �rea y principales conceptos
que permitan definir el problema. Aqu� tambi�n debe explicar claramente su aporte.
 Debe haber un p�rrafo explicando la estructura del escrito.
Extensi�n sugerida  es de \textbf{2 p�ginas}.


\chapter{Marco Referencial}
\label{marco}


Los contenidos del informe se definen seg�n el tipo de
trabajo (desarrollo, investigaci�n u otro). El presente documento expone
los contenidos m�nimos exigidos para los cap�tulos \emph{Marco
Referencial} y \emph{Definici�n del problema}. Estos cap�tulos son
obligatorios. No obstante, es el profesor gu�a qui�n debe aprobar la
organizaci�n definitiva de cada uno de ellos seg�n el inter�s del
trabajo. Se debe resguardar la calidad y confiabilidad de las
fuentes bibliogr�ficas. (Este es un ejemplo de como referenciar \cite{khan2014,agrawal94,beeferman00,bharat97}. M\'as ejemplos \cite{dawson}. Para m\'as detalle, revise el archivo \textit{template.bib})

\newpage
\section{Trabajos de T�tulo de Desarrollo}

\subsection{Marco conceptual}

\subsubsection{Definici�n del �rea del problema} \label{contexto}
\vspace{2mm}
\normalsize
Situar el problema en un �rea espec�fica del conocimiento. definir
terminolog�a b�sica del �rea. Referenciar trabajos y resultados
fundamentales. Extensi�n sugerida de \textbf{4 p�ginas}.

\vspace{2mm}

\subsubsection{T�cnicas y herramientas existentes}
\label{tec}

\vspace{1mm}

\normalsize

Revisar sistemas similares, herramientas relacionadas y/o proyectos
relacionados a su trabajo. Resuma las contribuciones e ideas
centrales de cada herramienta relacionada en no m�s de media p�gina
por cada una. Su extensi�n m�xima es de \textbf{7 p�ginas}.

\vspace{2mm}

\subsubsection{Comparaci�n entre ellas} \label{comp}

\vspace{1mm}

\normalsize

Defina y priorice criterios de comparaci�n seg�n el inter�s del
problema. Incluya un cuadro comparativo de las herramientas indicando
su idea central, fortalezas y debilidades. Si existe una
categorizaci�n de ellas, debe incluirla en esta secci�n. Su
extensi�n m�xima es de \textbf{2 p�ginas}.

\vspace{3mm}
\section{Trabajos de T�tulo de Investigaci�n}


\subsection{Marco Referencial}
\label{marco}

\subsubsection{Definiciones}

\vspace{1mm}

\normalsize

Incluya definiciones de conceptos y terminolog�a b�sica del �rea.
S�lo incluya lo necesario para despu�s poder presentar su problema.
Referencie autores con contribuciones originales y relevantes al
�rea. Su extensi�n m�xima es de \textbf{5 p�ginas}.

\vspace{2mm}

\subsubsection{Estado del arte}

\vspace{1mm}

\normalsize

Revise la literatura e incluya trabajos relacionados. Referencia
s�lo aquellos trabajos relevantes a su problema. Clasifique los
trabajos. Puede usar una clasificaci�n estandar de su �rea. Su
extensi�n m�xima es de \textbf{10 p�ginas}.

\vspace{2mm}

\subsubsection{Comparaci�n entre trabajos relacionados}

\vspace{1mm}

\normalsize

Incluya un cuadro comparativo de los trabajos relacionados
definiendo fortalezas y debilidades de cada uno. Su extensi�n m�xima
es de \textbf{3 p�ginas}.

\vspace{3mm}
\normalsize
\chapter{Definici�n del Problema y An�lisis}
\section{Trabajos de T�tulo de Desarrollo}

\label{definicion}
\subsection{Definici�n del problema} \label{prob}

\vspace{1mm}

\normalsize

Redefina el problema que present� en su propuesta incluyendo
\textbf{todas} las observaciones que se le hicieron tanto al informe
escrito como a la presentaci�n oral que realiz�. Precise prop�sito y
metodolog�a. Incluya las siguientes secciones.

\vspace{1mm}

\subsection{Sistema actual}

\vspace{1mm}

\normalsize

En esta secci�n debe entregarse una s�ntesis de la definici�n y especificaci�n de requerimientos. S�lo deben incluirse en �l, los modelos principales. El detalle de los modelos debe entregarse en anexos. Al especificar los requerimientos debe verificar que estos sean necesarios, no ambiguos, trazables, factibles de ser medidos y probados, completos, consistentes y deben estar debidamente priorizados.

Descripci�n del sistema actual: �mbito, prop�sito, objetivos, usuarios, modos de operaci�n, etc. Su extensi�n m�xima es de \textbf{3 p�ginas}.

\subsection{Naturaleza del cambio}

\vspace{1mm}

\normalsize

Identifique, Describa la naturaleza del cambio que introducir�. Su extensi�n
m�xima es de \textbf{2 p�ginas}.

\vspace{1mm}

\subsection{Especificaci�n de requerimientos}

\vspace{1mm}

\normalsize

Descripci�n formal de la soluci�n propuesta (Modos de operaci�n y  An�lisis). Metodolog�a de desarrollo.  Su extensi�n m�xima es de \textbf{15 p�ginas}.


\section{Trabajos de T�tulo de Investigaci�n}
\subsubsection{Definici�n del problema}

\vspace{1mm}

\normalsize

Redefina el problema que present� en su propuesta incluyendo
\textbf{todas} las observaciones que se le hicieron tanto al informe
escrito como a la presentaci�n oral que realiz�. Precise prop�sito y
metodolog�a. Incluya las siguientes secciones.

\vspace{1mm}

\subsubsection{Definici�n del problema}

\vspace{1mm}

\normalsize

Defina formalmente su problema. Defina expl�citamente hip�tesis o lineamientos, preguntas de investigaci�n, y metodolog�a
de investigaci�n.  Use notaci�n matem�tica de ser
necesario. Su extensi�n m�xima es de \textbf{3 p�ginas}.

\vspace{1mm}

\section{Soluci�n propuesta}

\vspace{1mm}

\normalsize

Describa la soluci�n propuesta. Incluya objetivos generales y
espec�ficos. Su extensi�n m�xima es de \textbf{3 paginas}.





\chapter{Dise�o}
\label{capdiseno}

\section{Introducci�n}

El Informe 3 equivale al cap�tulo de Dise�o de su Trabajo de T�tulo.
 En este documento se describen los contenidos m�nimos exigidos para el Informe 3 y se sugiere la extensi�n (en cantidad de p�ginas) de cada t�pico.
Los contenidos  se definen seg�n el tipo de trabajo que se est� realizando: desarrollo de SW, investigaci�n,  u otro.
 Sin embargo, es el profesor gu�a qui�n debe aprobar su organizaci�n definitiva.
 Tambi�n se debe resguardar la calidad y confiabilidad de las fuentes bibliogr�ficas.

\section{Dise�o Trabajos de T�tulo de Desarrollo de SW}  \label{diseno}
 En este cap�tulo s e describe los contenidos del Cap�tulo de dise�o para los TT de desarrollo de software.
 
 Obs.: Resguardar la trazabilidad y consistencia entre los modelos. Justifique las
decisiones de dise�o.

\subsection{Dise�o arquitect�nico} \label{disenoarq}
Defina (si corresponde) los patrones de dise~no que usar�. Incluya modelo de estructura
del sistema el cual debe reflejar el tipo de arquitectura especifica (cliente-servidor,
3 capas, SOA). Cada m�dulo debe estar trazado con respecto a los subsistemas identificados en el modelo de estructura.
De ser necesario incluya modelo de control.
Extensi�n m�xima sugerida 5 p�ginas.


\subsection{Dise�o de interfaz} \label{disenoint}
Sobre la base del perfil de usuario debe seleccionar el estilo de interacci�n, definir
los objetivos de facilidad de uso, determinar las pautas de estilo. Incluya en anexo el
modelo de navegaci�on. Extensi�n m�xima sugerida 5 p�ginas.

\subsection{Dise�o l�gico} \label{disenolog}
De acuerdo con  la metodolog�a seleccionada, especifique los modelos de dise~no requerido,
por ejemplo para orientaci�n a objeto casos de uso reales (en anexo), diagramas de
colaboraci�n (en anexo) y diagramas de clases. Extensi�n m�xima sugerida 15 p�ginas.


\subsection{Dise�o de datos}  \label{disenodat}
A partir de cada subsistema (consistente con el modelo de estructura del sistema),
definir una componente ER con notaci�n Bachmann. Defina claves candidatas, entidades,
cardinalidades y atributos. Normalice y justifique la normalizaci�n. Integre los
componentes ER en un modelo de datos l�gico interno global Bachmann identificando
claves primarias, for�neas, atributos, tipos de relaciones y cardinalidades. Extensi�n m�xima sugerida 5 p�ginas.


 \subsection{Dise�o de pruebas}

  Debe definir cuidadosamente el objetivo y como realizar� las pruebas de cada parte de su desarrollo:  Pruebas de requerimientos, Pruebas de an�lisis,  Pruebas de dise�o, Pruebas de unidad, Pruebas de integraci�n. Pruebas de sistema. Pruebas de aceptaci�n del usuario, entre otras.


\subsection{Conclusiones}  \label{conclusiones}
Incluya an�lisis cr�tico sobre  pertinencia del problema, soluci�n propuesta, proyecciones
y estado de avance. Extensi�n m�xima sugerida 2 p�ginas.




\section{Dise�o para Trabajos de T�tulo de Investigaci�n}
\label{diseno}
En este cap�tulo s e describe los contenidos del Cap�tulo de dise�o para los TT de Investigaci�n
\subsection{Dise�o de la soluci�n} \label{disenosol}
Defina brevemente el contexto de la soluci�n propuesta. Luego defina detalladamente
la soluci�n. Para esto debe incluir modelos o diagramas descriptivos globales
y luego incluir informaci�n detallada sobre cada m�dulo. En el caso de definici�n de
modelos, se deben incluir variables a considerar y la relaci�n entre estas. Justifique
su dise�o sea riguroso(a).

 Si su soluci�n incluye el desarrollo de algun SW tambi�n debe realizar el dise�o del mismo, de acuerdo a la pauta anterior.
   Extensi�n m�xima sugerida de 15 p�ginas.


\subsection{Dise�o de experimentos} \label{disenoexp}
Defina los  experimentos que debe realizar para responder sus preguntas de investigaci�n.
Cada experimento  debe precisar objetivo, escenarios posibles,
variables involucradas, medidas con las que se trabajar�, pre-experimentos si es
necesario, relaci�n entre las variables (hip�tesis), herramientas y pasos del an�lisis.

 En el caso que los experimentos est�n encadenados, incluya un diagrama de causalidad de
experimentos.  Extensi�n m�xima sugerida de 15 p�ginas.

Obs: lea cuidadosamente los documentos sobre dise�o de experimentos \cite{inv,dawson,fundibeq,extracto_dawson}. Siga la pauta mostrada en las ppt de dise�o de experimentos \cite{extracto_dawson_ppt}.

\subsection{Conclusiones} \label{conclusiones}
Incluya an�lisis cr�tico, pertinencia del problema, soluci�n propuesta, proyecciones
y estado de avance. Extensi�n m�xima sugerida  2 p�ginas.


\section{Dise�o para otros tipos  Trabajos de T�tulo }

 Para TT distintos a los explicados anteriormente deber�n considerar las secciones que  les acomoden, en  acuerdo con su profesor gu�a.
  Si su soluci�n incluye el desarrollo de algun SW tambi�n debe realizar el dise�o del mismo, de acuerdo a la pauta anterior.
 

%\include{capitulo3}
%\include{capitulo4}
%\include{capitulo5}

%\appendix

%\include{appendix1}
%\include{appendix2}
%\include{appendix3}

\bibliographystyle{plain}
\bibliography{template,informe}
%\bibliography{informe}

\end{document} 